\documentclass{article}

\usepackage{tabularx} 		% for "tabularx" section
\usepackage[table]{xcolor} 	% for "\cellcolor"
\usepackage{booktabs}		% for the "\toprule" and "\bottomrule" commands
\usepackage{amssymb} 		% for the "\checkmark"
\usepackage{graphicx}		% for the "\rotatebox" command

\usepackage{times}		% to use Adobe Times font (not mandatory)

\begin{document}

\section{A Simple Related Work Table}

\begin{table}[!htbp]
	\centering

	\footnotesize

	\newcommand\cm{\checkmark}

	\definecolor{RWGray}{gray}{0.85}

	\begin{tabularx}{\linewidth}{p{7cm}>{\columncolor{RWGray}}XX>{\columncolor{RWGray}}XX>{\columncolor{RWGray}}XX}
		\toprule	
																	&
		\rotatebox[origin=l]{90}{A feature of your work}			&
		\rotatebox[origin=l]{90}{Another one} 						&
		\rotatebox[origin=l]{90}{A nice feature} 					&
		\rotatebox[origin=l]{90}{A better one} 						&
		\rotatebox[origin=l]{90}{A really big feature description} 	&
		\rotatebox[origin=l]{90}{The last feature}					\\

		\hline
		Some Related Work	& \cm 	&       &		&		& \cm 	& 		\\
		\hline
		Some Related Work	& \cm   &       &		&       & \cm   &       \\
		\hline
		Some Related Work	& \cm   &       &       &       & \cm   &       \\
		\hline
		Some Related Work	& \cm   &       &       &       & \cm   &       \\
		\hline
		Some Related Work	&       & \cm   &       & \cm   & \cm   & \cm   \\
		\hline
		Some Related Work	& \cm   & \cm   & \cm   &    	&   	& \cm   \\
		\hline
		Your Work			& \cm   & \cm   & \cm   & \cm   & \cm   & \cm   \\
	\bottomrule
	\end{tabularx}
	\caption{State of the Art Table.}
	\label{tbl:state-of-art}
\end{table}

\end{document}
